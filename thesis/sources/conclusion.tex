With the evergrowing popularity of delivery services, customers do no longer settle for a mediocre experience. This is especially true in the food delivery market, where customers demand the fastest possible delivery time for the lowest possible cost. To be able to compete in this market, companies depend upon efficient planning and order dispatching, which has to be done in real time given the dynamic environment of gastronomy.

This study contributes to the area of vehicle routing problems. We have devised a method to solve the PDPTW in the food delivery settings that uses several novel approaches adopted from the recent studies. The usefulness of our work lies in the integration of the planning algorithm to the GoDeliver system and its successful deployment to the real customers.

The proposed algorithm is adopted mainly from the research of DARP by Masmoudi et al. (2020) \cite{Masmoudi2020} and adapted to the PDPTW by introducing different constraints and objectives, as well as altered operators and custom parameters. The algorithm was implemented in Go language and benchmarked against insertion heuristics and Google's ORTools.

We have obtained encouraging results demonstrating that the proposed algorithm is able to solve large-scale instances emanating from real-life usecases. The results also emphasize the importance of the combination of intensification and diversification mechanisms within the search.

An important issue to address in future is regarding the performance of the algorithm. Additional performance optimization and parallelization is needed to achieve faster and more predictable execution times, which would make it easier to scale the algorithm to even larger instances. Moreover, further parameter tuning could help to find overall better solutions.

Consumer habits have been increasingly shifting towards online and this process has been accelerated by the global 2020 COVID-19 outbreak. Online sales grows by 44\,\% year by year in the US alone \cite{ecommerce-growth}. In addition, these changes in customer behavior are likely to have lasting effects \cite{UNCTAD}. 

The category of logistics that has recently increased in popularity is food delivery \cite{food-delivery-statista}. The advantages of food delivery services are immense, especially during the COVID-19 pandemic, as they facilitate customer access to ready-to-eat meals and enable restaurants and other food providers to keep operating in times of government restrictions \cite{food-delivery}.

However, the planning of food delivery is a complex problem due to the dynamic setting that is subject to constant change, and only a few global food delivery platforms are able to dispatch orders and plan routes efficiently to lower the delivery cost while meeting the customer expectations. In contrast, local restaurants struggle to provide customers with a good logistics experience because they lack the necessary tools. As a result, they must turn to these global food delivery platforms, which often come with higher cost and lower quality of service.

GoDeliver\footnote{\url{https://www.godeliver.co/}} is a software solution that aims to solve this issue. It provides local businesses with a tool for managing their delivery fleets, which includes automatic order dispatching, and route planning. However, larger businesses with more than one restaurant, such as fast food chains, require more sophisticated planning algorithms that are able to plan near-optimal routes for tens of drivers consisting of hundreds of customers in a reasonable time, given the dynamic environment. As of today, the planning algorithms used by GoDeliver are not ready for such big instances in a dynamic setting.

The problem of finding optimal routes for a fleet of vehicles is a so-called \emph{vehicle routing problem} which has been extensively studied in the literature over the past 60 years. Several promising algorithms have been proposed for a wide range of variants of this problem, including people transportation, and delivery of goods. While most studies focus on basic variants of the vehicle routing problem, this thesis explores the online pickup and delivery problem with time windows with the emphasis on food delivery in which the customer experience is the most important aspect. Therefore, the primary objective is to deliver the food fresh and with the lowest delay possible. Furthermore, because food delivery is a low-margin business, it is also important to keep in mind the cost of delivery to make profit.

The main goal of this thesis is to propose and implement a planning algorithm based on the state-of-the-art research to solve the online pickup and delivery problem with time windows that enables to tackle the real-world use cases of food delivery. The secondary goal is to integrate this algorithm with GoDeliver such that it can be used in production settings. The results of this study will greatly contribute to improving the GoDeliver system by equipping it with faster and stronger planning power, which will be able to serve larger businesses and contribute to the revolution of last-mile logistics.

The experimental results suggest that our algorithm finds overall better solutions on larger instances compared to several baselines. In addition, it seems to find a good tradeoff between minimizing the delays and the cost of delivery. On the other hand, the proposed algorithm needs considerably longer running time.

This thesis is organized as follows: Chapter \ref{chapter:literature} provides an overview of the vehicle routing problem and presents existing solution methods from the literature. Chapter \ref{chapter:problem} defines the problem of planning of food delivery and discusses its specifics. It also explains the need for an advanced planning algorithm in the GoDeliver system pipeline and defines the requirements for this algorithm. Chapter \ref{chapter:methodology} presents the methodology and chapter \ref{chapter:results} shows the computational experiments of the proposed algorithm on our evaluation dataset.
